\section*{Chapter 1}

\begin{itemize}
\item [(1)]
\tgl
		{Nae -ga yeoseot sal jeok -e han -beon -eun ``cheheom -ha -n iyagi" -laneun jemok -ui, wonsilim -e gwanha -n chaek -eseo}
		{내 -가 여섯 살 적 -에 한 -번 -은 ``체험 -하 -ㄴ 이야기” -라는 제목 -의, 원시림 -에 관하 -ㄴ 책 -에서}
		{\textsc{1.sg} \textsc{nom} six age time \textsc{loc} one occasion \textsc{top}	experience \textsc{vbz} \textsc{rel.pst} story \textsc{quot} title \textsc{gen} primeval\_forest \textsc{loc} be\_about \textsc{rel.pst} book \textsc{loc}}
		{Once when I was six-year-old, from a book about primeval forests with the title ``Tales experienced,"}
		
\tgl
		{gimaki -n geurim hana -reul bo -n jeog -i it -da.}
		{기막히 -ㄴ 그림 하나 -를 보 -ㄴ 적 -이 있 -다.}
		{amazing \textsc{rel.pst} picture one \textsc{acc} see \textsc{rel.pst} time \textsc{nom} \textsc{exist} \textsc{ind}}
		{I saw an amazing drawing.}

\item [(2)]
\tgl
		{Maengsu -reul jibeosamki -go it -neun boa gureongi geurim -i -eot -da}
		{맹수 -를 집어삼키 -고 있 -는 보아 구렁이 그림 -이 -었 -다.}
		{beast \textsc{acc} swallow \textsc{prog} \textsc{exist} \textsc{rel.prs} boa python picture \textsc{cop} \textsc{pst} \textsc{ind}}
		{It was a drawing of boa constrictor swallowing a whole beast.}

\item [(3)]
\tgl
		{Geugeos -eul omgi -eo geuri -myeon iro -ta}
		{그것 -을 옮기 -어 그리 -면 이렇 -다.}
		{that \textsc{acc} carry \textsc{conn} draw \textsc{sbjv} s\_such \textsc{ind}}
		{If one copies over that, it's as follows.}

\item [(4)]
\tgl
		{Geu chaek -e -neun iro -ke ssui -eo iss -eoss -da}
		{그 책 -에 -는 이렇 -게 씌 -어 있 -었 -다.}
		{\textsc{def} book \textsc{loc} \textsc{top} as\_such \textsc{comp} write \textsc{conn} \textsc{exist} \textsc{pst} \textsc{ind}}
		{It said this in the book.}

\item [(5)]
\tgl
		{Boa gureongi -neun meogi -reul ssip -ji -do an -ko tongjjae -ro jibeosamki -nda}
		{``보아 구렁이 -는 먹이 -를 씹 -지 -도 않 -고 통째 -로 집어삼키 -ㄴ다.}
		{boa python \textsc{top} prey \textsc{acc} chew \textsc{conn} even \textsc{neg} \textsc{conj} whole \textsc{adv} swallow \textsc{ind}}
		{``Boa constrictors swallow their preys even without chewing.}

\item [(6)]
\tgl
		{Geurigo -neun kkomjjak -do ha -ji mota -go yeosot dal dongan jam -eul ja -myeonseo geugeot -eul sohwa -shiki -nda}
		{그리고 -는 꼼짝 -도 하 -지 못하 -고 여섯 달 동안 잠 -을 자 -면서 그것 -을 소화 -시키 -ㄴ다."}
		{then \textsc{top} still even do \textsc{conn} cannot\_do \textsc{conj} six month during sleep \textsc{acc} sleep while that \textsc{acc} digest \textsc{pass} \textsc{ind}}
		{Then, staying still, they digest it while sleeping for six months."}

\item [(7)]
\tgl
		{Na -neun geuraeseo millim sok -eseo -ui moheom -e daeha -e hancham saenggakha -e bo -go na -n ggeut -e}
		{나 -는 그래서 밀림 속 -에서 -의 모험 -에 대하 -아 한참 생각하 -아 보 -고 나 -ㄴ 끝 -에}
		{\textsc{1.sg} \textsc{top} therefore jungle inside \textsc{loc} \textsc{gen} adventure \textsc{loc} be\_about \textsc{conn} for\_a\_while think \textsc{conn} try \textsc{conj} happen \textsc{rel.pst} end \textsc{loc}}
		{So I, after pondering about adventures in jungles for a while,}
		
\tgl
		{saekyeonpil -eul gaji -go nae nareum -daero nae saengae cheot beonjjae geurim -eul geuri -eo -bo -at -da}
		{색연필 -을 가지 -고 내 나름 -대로 내 생애 첫 번째 그림 -을 그리 -어 -보 -았 -다.}
		{colored\_pencil \textsc{acc} have \textsc{conj} \textsc{1.poss} means as \textsc{1.poss} life first time picture \textsc{acc} draw \textsc{conn} try \textsc{pst} \textsc{ind}}
		{drew a picture for the first time in my life with colored pencils in my own ways.}

\item [(8)]
\tgl
		{Na -ui geurim je -il -ho -i -eot -da}
		{나 -의 그림 제 -1 -호 -이 -었 -다.}
		{\textsc{1.sg} \textsc{gen} picture \textsc{ord} one number \textsc{cop} \textsc{pst} \textsc{ind}}
		{It was my first drawing.}

\item [(9)]
\tgl
		{Geugeot -eun ireon geurim -i -eot -da}
		{그것 -은 이런 그림 -이 -었 -다.}
		{that \textsc{top} as\_follows picture \textsc{cop} \textsc{pst} \textsc{ind}}
		{The drawing was like this.}

\item [(10)]
\tgl
		{Na -neun geu geoljakpum -eul eoreun -deul -ege boy -eo ju -myeonseo nae geurim -i museop -ji an -neunyago mul -eot -da.}
		{나 -는 그 걸작품 -을 어른 -들 -에게 보이 -어 주 -면서 내 그림 -이 무섭 -지 않 -느냐고 물 -었 -다.}
		{\textsc{1.sg} \textsc{top} \textsc{def} masterpiece \textsc{acc} adult \textsc{pl} \textsc{dat} show \textsc{conn} give \textsc{cvb} \textsc{1.sg.gen} picture \textsc{nom} scary \textsc{conn} \textsc{neg} \textsc{quot} ask \textsc{pst} \textsc{ind}}
		{I showed the masterpiece to grown-ups and asked them if my drawing was terrifying.}

\item [(11)]
\tgl
		{Geudeul -eun ``moja -ga mwo -ga museop -daneun geo -ni?" hago daedap -ha -et -da.}
		{그들 -은 ``모자 -가 뭐 -가 무섭 -다는 거 -니?" 하고 대답 -하 -았 -다.}
		{\textsc{3.pl} \textsc{top} hat \textsc{emp} what \textsc{nom} scary \textsc{comp} thing \textsc{q} \textsc{quot} answer \textsc{vbz} \textsc{pst} \textsc{ind}}
		{They answered: "what is so terrifying about a hat?"}

\item [(12)]
\tgl
		{Nae geurim -eun moja -reul geuri -n ge ani -eot -da.}
		{내 그림 -은 모자 -를 그리 -ㄴ 게 아니 -었 -다.}
		{\textsc{1.sg.gen} picture \textsc{top} hat \textsc{acc} draw \textsc{rel.pst} thing.\textsc{nom} \textsc{cop.neg} \textsc{pst} \textsc{ind}}
		{My drawing was not of a hat.}

\item [(13)]
\tgl
		{Geugeot -eun kokkiri -reul sohwa -siki -go it -neun boa gureongi -yeot -da.}
		{그것 -은 코끼리 -를 소화 -시키 -고 있 -는 보아 구렁이 -였 -다.}
		{that \textsc{top} elephant \textsc{acc} digest \textsc{caus} \textsc{prog} \textsc{exist} \textsc{rel.prs} boa python \textsc{cop.pst} \textsc{ind}}
		{It was a boa constrictor digesting an elephant.}

\item [(14)]
\tgl
		{Geuraeseo na -neun eoreun -deul -i alabo -l su it -dorok boa gureongi -ui sok -eul geury -eot -da.}
		{그래서 나 -는 어른 -들 -이 알아보 -ㄹ 수 있 -도록 보아 구렁이 -의 속 -을 그리 -었 -다.}
		{so \textsc{1.sg} \textsc{top} adult \textsc{pl} \textsc{nom} recognize \textsc{rel} means \textsc{exist} \textsc{comp} boa python \textsc{gen} inside \textsc{acc} draw \textsc{pst} \textsc{ind}}
		{So I drew the inside of the boa constrictor in order that grown-ups can recognize it.}

\item [(15)]
\tgl
		{Eoreun -deul -eun eonjena seolmyeong -eul ha -e -ju -eoyaman ha -nda.}
		{어른 -들 -은 언제나 설명 -을 하 -아 -주 -어야만 하 -ㄴ다.}
		{adult \textsc{pl} \textsc{top} always explanation \textsc{acc} do \textsc{conn} give \textsc{oblig} do \textsc{ind}}
		{Grown-ups always need explanations.}

\item [(16)]
\tgl
		{Na -ui geurim je -2 -ho -neun ireoha -et -da.}
		{나 -의 그림 제 -2 -호 -는 이러하 -었 -다.}
		{\textsc{1.sg} \textsc{gen} picture \textsc{ord} one number \textsc{top} as\_such \textsc{pst} \textsc{ind}}
		{My drawing no. 2 was like this.}

\item [(17)]
\tgl
		{Eoreun -deul -eun sok -i boi -geona boi -ji an -keona ha -neun boa gureongi -ui geurim -deul -eun jibeochiu -go}
		{어른 -들 -은 속 -이 보이 -거나 보이 -지 않 -거나 하 -는 보아 구렁이 -의 그림 -들 -은 집어치우 -고}
		{adult	\textsc{pl}	\textsc{top}	inside	\textsc{nom}	be\_seen	\textsc{disj}	be\_seen	\textsc{conn}	\textsc{neg}	\textsc{disj}	say	\textsc{rel.prs}	boa	python	\textsc{gen}	picture	\textsc{pl}	\textsc{top}	quit	\textsc{conj}}
		{Grown-ups, telling me to throw away those drawings of boa constrictors with visible interior or else,}

\tgl
		{charari jiri, yeoksa, gyesan, geurigo munbeop jjok -e gwansim -eul gajy -eo -bo -neun ge jo -eul geot -i -rago chunggo -ha -e ju -eot -da.}
		{차라리 지리, 역사, 계산, 그리고 문법 쪽 -에 관심 -을 가지 -어 -보 -는 게 좋 -을 것 -이라고 충고 -하 -아 주 -었 -다.}
		{rather	geography	history	arithmetics	\textsc{conj}	grammar	side	\textsc{loc}	interest	\textsc{acc}	have	\textsc{conn}	try	\textsc{rel.prs}	thing.\textsc{nom}	good	\textsc{rel}	thing	\textsc{comp}	advice	\textsc{vbz}	\textsc{conn}	give	\textsc{pst}	\textsc{ind}}
		{advised that taking interest in geopgraphy, history, arithmetics and grammar instead would rather be better.}

\item [(18)]
\tgl
		{Geuraeseo na -neun yeoseot sal jeok -e hwaga -raneun meotji -n jikeop -eul pogi -ha -e beory -eot -da.}
		{그래서 나 -는 여섯 살 적 -에 화가 -라는 멋지 -ㄴ 직업 -을 포기 -하 -아 버리 -었 -다.}
		{so	\textsc{1.sg}	\textsc{top}	six	age	time	\textsc{loc} 	painter	\textsc{quot}	nice	\textsc{rel.pst}	occupation	\textsc{acc}	giving\_up	\textsc{vbz}	\textsc{conn}	end\_up	\textsc{pst}	\textsc{ind}}
		{So I, at the age of six, gave up such a cool job that is a painter.}

\item [(19)]
\tgl
		{Nae geurim je -il -ho -wa je -i -ho -ga seonggong -eul geodu -ji motha -n de naksim -ha -e beory -eot -deon geot -i -da.}
		{내 그림 제 -1 -호 -와 제 -2 -호 -가 성공 -을 거두 -지 못하 -ㄴ 데 낙심 -하 -아 버리 -었 -던 것 -이 -다.}
		{\textsc{1.sg.gen}	picture	\textsc{ord}	one	number	\textsc{com} \textsc{ord}	two	number	\textsc{nom}	success	\textsc{acc}	achieve	\textsc{conn}	fail	\textsc{rel.pst}	place	disappointment	\textsc{vbz}	\textsc{conn}	end\_up	\textsc{pst}	\textsc{rel.ipfv}	thing	\textsc{cop}	\textsc{ind}}
		{I had been disappointed at the fact that my pictures no. 1 and 2 failed to succeed.}

\item [(20)]
\tgl
		{Eoreun -deul -eun eonjena seuseuro -neun amugeot -do ihae -ha -ji motha -nda.}
		{어른 -들 -은 언제나 스스로 -는 아무것 -도 이해 -하 -지 못하 -ㄴ다.}
		{adult	\textsc{pl}	\textsc{top}	always	by\_oneself	\textsc{top}	anything	\textsc{top}	understanding	\textsc{vbz}	\textsc{conn}	cannot	\textsc{ind}}
		{Grown-ups always are not capable of understanding by themselves.}

\item [(21)]
\tgl
		{Jakkujakku seolmyeon -geul ha -e -ju -eoya ha -ni maek -ppaji -neun noreut -i ani -l su eop -da.}
		{자꾸자꾸 설명 -을 하 -아 -주 -어야 하 -니 맥 -빠지 -는 노릇 -이 아니 -ㄹ 수 없 -다.}
		{again\_and\_again	explanation	\textsc{acc}	do	\textsc{conn}	give	\textsc{oblig}	do	\textsc{comp}	energy	drain	\textsc{rel}	situation	\textsc{nom}	\textsc{cop.neg}	\textsc{rel}	means	\textsc{exist.neg}	\textsc{ind}}
		{As they need explanations again and again, there can't be a way it's not an exhausting situation.}

\item [(22)]
\tgl
		{Geuraeseo dareu -n jikeop -eul seontaek -ha -ji an -eul su eop -ge doe -n na -neun bihaenggi jojong -ha -neun beop -eul baewu -ot -da.}
		{그래서 다르 -ㄴ 직업 -을 선택 -하 -지 않 -을 수 없 -게 되 -ㄴ 나 -는 비행기 조종 -하 -는 법 -을 배우 -었 -다.}
		{so	different	\textsc{rel.pst}	occupation	\textsc{acc}	choice	\textsc{vbz}	\textsc{conn}	\textsc{neg}	\textsc{rel}	means	\textsc{exist.neg}	\textsc{comp}	become	\textsc{rel.pst}	\textsc{1.sg}	\textsc{top}	airplane	piloting	\textsc{vbz}	\textsc{rel.prs}	method	\textsc{acc}	learn	\textsc{pst} \textsc{ind}}
		{So, having no other option than choosing another job, I learned how to pilot a plane.}

\item [(23)]
\tgl
		{Segye -ui yeogijeogi geoui an ga -a -bo -n de eops -i na -neun naladany -eot -da.}
		{세계 -의 여기저기 거의 안 가 -아 -보 -ㄴ 데 없 -이 나 -는 날아다니 -었 -다.}
		{world	\textsc{gen}	here\_and\_there	almost	\textsc{neg}	go	\textsc{conn}	try	\textsc{rel.pst}	place	\textsc{exist.neg}	\textsc{adv}	\textsc{1.sg}	\textsc{top}	fly\_around	\textsc{pst} \textsc{ind}}
		{I flew around here and there of the world, leaving almost no places unvisited.}

\item [(24)]
\tgl
		{Geureoni jiri -neun jeongmalro man -eun doum -eul ju -n sem -i -eot -da.}
		{그러니 지리 -는 정말로 많 -은 도움 -을 주 -ㄴ 셈 -이 -었 -다.}
		{therefore	geography	\textsc{top}	truly	much	\textsc{rel.pst}	help	\textsc{acc}	give	\textsc{rel.pst}	result	\textsc{cop}	\textsc{pst}	\textsc{ind}}
		{}

\item [(25)]
\tgl
		{Han -beon seuljjeok bo -go -do jungguk -gwa aerijona -reul na -neun gubyeol -ha -l su iss -eot -deon geos -i -da.}
		{한 -번 슬쩍 보 -고 -도 중국 -과 애리조나 -를 나 -는 구별 -하 -ㄹ 수 있 -었 -던 것 -이 -다.}
		{one	time	lightly	see	\textsc{conj}	even	China	\textsc{com}	Arizona	\textsc{acc}	\textsc{1.sg}	\textsc{top}	distinction	\textsc{vbz}	\textsc{rel}	means	\textsc{exist}	\textsc{pst}	\textsc{rel.ipfv}	thing	\textsc{cop}	\textsc{ind}}
		{I was even able to give a quick glance and distinguish China from Arizona.}

\item [(26)]
\tgl
		{Geugeos -eun bam -e gir -eul ir -eoss -eul ttae aju yuyong -ha -n ir -i -da.}
		{그것 -은 밤 -에 길 -을 잃 -었 -을 때 아주 유용 -하 -ㄴ 일 -이 -다.}
		{that	\textsc{top}	night	\textsc{loc}	way	\textsc{acc}	lose	\textsc{pst}	\textsc{rel}	time	very	usefulness	\textsc{vbz}	\textsc{rel.pst}	thing	\textsc{cop}	\textsc{ind}}
		{That is a very helpful thing when lost in the night.}

\item [(27)]
\tgl
		{Na -neun geurihayeo ilsaeng doan sueobsi man -eun jeomjan -eun saram -deul -gwa suman -eun jeopchog -eul gajyeo -wat -da.}
		{나 -는 그리하여 일생 동안 수없이 많 -은 점잖 -은 사람 -들 -과 수많 -은 접촉 -을 가져오 -았 -다.}
		{\textsc{1.sg}	\textsc{top}	so	life	course	countlessly	many	\textsc{rel.pst}	gentle	\textsc{rel.pst}	person	\textsc{pl}	\textsc{com}	countless	\textsc{rel.pst}	contact	\textsc{acc}	bring	\textsc{pst}	\textsc{ind}}
		{So I have been having encounters with many countless gentle people in the course of my life.}

\item [(28)]
\tgl
		{Eoreun -deul teum -eseo mani sarao -n geos -i -da.}
		{어른 -들 틈 -에서 많이 살아오 -ㄴ 것 -이 -다.}
		{adult	\textsc{pl}	gap	\textsc{loc}	plenty	live	\textsc{rel.pst}	thing	\textsc{cop}	\textsc{ind}}
		{I have lived among grown-ups for long.}

\pagebreak

\item [(29)]
\tgl
		{Na -neun gakkai -seo geu -deur -eul bo -l su iss -eot -da.}
		{나 -는 가까이 -서 그 -들 -을 보 -ㄹ 수 있 -었 -다.}
		{\textsc{1.sg}	\textsc{top}	nearby	\textsc{loc}	\textsc{3}	\textsc{pl}	\textsc{acc}	see	\textsc{rel}	means	\textsc{exist}	\textsc{pst}	\textsc{ind}}
		{I was able to observe them up close.}

\item [(30)]
\tgl
		{Geureo -tago ha -eseo geu -deur -e daeha -n nae saenggag -i naaji -n geon eobs -eot -da.}
		{그렇 -다고 하 -아서 그 -들 -에 대하 -ㄴ 내 생각 -이 나아지 -ㄴ 건 없 -었 -다.}
		{as\_such	\textsc{conn}	say	\textsc{conn}	\textsc{3}	\textsc{pl}	\textsc{loc}	be\_about	\textsc{rel.pst}	\textsc{1.sg.gen}	thought	\textsc{nom}	improve	\textsc{rel.pst}	thing.\textsc{top}	\textsc{exist.neg}	\textsc{pst}	\textsc{ind}}
		{Having said that, it didn't improve my impression of them at all.}

\item [(31)]
\tgl
		{Jogeum chongmyeong -ha -e boi -neun saram -eul manna -l ttae -myeon}
		{조금 총명 -하 -아 보이 -는 사람 -을 만나 -ㄹ 때 -면}
		{a\_little	bright	\textsc{vbz}	\textsc{conn}	seem	\textsc{rel.prs}	person	\textsc{acc}	meet	\textsc{rel}	time	\textsc{sbjv}}
		{Whenever I met someone who looks somewhat bright,}
		
\tgl
		{na -neun neul ganjik -ha -e o -go it -deon ye -ui na -ui geurim je -il -ho -reul gaji -go geu saram -eul siheom -ha -e bo -goneun haet -da.}
		{나 -는 늘 간직 -하 -아 오 -고 있 -던 예 -의 나 -의 그림 제 -1 -호 -를 가지 -고 그 사람 -을 시험 -하 -아 보 -고는 하 -았 -다.}
		{\textsc{1.sg}	\textsc{top}	always	keep	\textsc{vbz}	\textsc{conn}	come	\textsc{conn}	\textsc{exist}	\textsc{rel.ipfv}	instance	\textsc{gen}	\textsc{1.sg}	\textsc{gen}	picture	\textsc{ord}	one	number	\textsc{acc}	have	\textsc{conj}	\textsc{def}	person	\textsc{acc}	test	\textsc{vbz}	\textsc{conn}	try	used\_to	\textsc{aux}	\textsc{pst}	\textsc{ind}}
		{I used to test him with the aforementioned drawing no. 1 of mine that I've always been keeping.}

\item [(32)]
\tgl
		{Geu saram -i jeongmalro mwol ihae -ha -l jul a -neun saram -i -nga al -go sip -eot -deon geos -i -da.}
		{그 사람 -이 정말로 뭘 이해 -하 -ㄹ 줄 알 -는 사람 -이 -ㄴ가 알 -고 싶 -었 -던 것 -이 -다.}
		{\textsc{def}	person	\textsc{nom}	truly	what.\textsc{acc}	understand	\textsc{vbz}	\textsc{rel}	means	know	\textsc{rel.prs}	person	\textsc{cop}	\textsc{q}	know	\textsc{conn}	want	\textsc{pst}	\textsc{rel.ipfv}	thing	\textsc{cop}	\textsc{ind}}
		{I wanted to know if he was someone who is capable of truly understanding stuffs.}

\item [(33)]
\tgl
		{Geureona eure geu saram -eun moja -raneun geos -i -eot -da.}
		{그러나 으레 그 사람 -은 모자 -라는 것 -이 -었 -다.}
		{however	always	\textsc{def}	person	\textsc{top}	hat	\textsc{quot}	thing	\textsc{cop}	\textsc{pst}	\textsc{ind}}
		{But that person would always say it was a hat.}

\item [(34)]
\tgl
		{Geureomyeon na -neun boa gureoi -do wonsirim -do byeol -deul -do geu -ege iyagi -ha -ji an -at -da.}
		{그러면 나 -는 보아 구렁이 -도 원시림 -도 별 -들 -도 그 -에게 이야기 -하 -지 않 -았 -다.}
		{then	\textsc{1.sg}	\textsc{top}	boa	python	\textsc{conj}	primeval\_forst	\textsc{conj}	star	\textsc{pl}	\textsc{conj}	\textsc{3.sg}	\textsc{dat}	story	\textsc{vbz}	\textsc{conn}	\textsc{neg}	\textsc{pst}	\textsc{ind}}
		{Then I would never talk to him about boa constrictors, primeval forests or stars.}

\item [(35)]
\tgl
		{Geu -ga ihae -ha -l su it -neun iyagi -reul ha -et -da.}
		{그 -가 이해 -하 -ㄹ 수 있 -는 이야기 -를 하 -았 -다.}
		{\textsc{3.sg}	\textsc{nom}	understanding	\textsc{vbz}	\textsc{rel}	means	\textsc{exist}	\textsc{rel.prs}	story	\textsc{acc}	tell	\textsc{pst}	\textsc{ind}}
		{I talked about things he could understand.}

\item [(36)]
\tgl
		{Beuriji -ni golpeu -ni jeongchi -ni nektai -ni ha -neun geot -deur -e daeha -e iyagi -ha -neun geos -i -da.}
		{브리지 -니 골프 -니 정치 -니 넥타이 -니 하 -는 것 -들 -에 대하 -아 이야기 -하 -는 것 -이 -다.}
		{bridge	\textsc{disj}	golf	\textsc{disj}	politics	\textsc{disj}	necktie	\textsc{disj}	say	\textsc{rel.prs}	thing	\textsc{pl}	\textsc{loc}	be\_about	\textsc{conn}	story	\textsc{vbz}	\textsc{rel.prs}	thing	\textsc{cop}	\textsc{ind}}
		{Talked about bridge, golf, politics, neckties and whatnot.}

\item [(37)]
\tgl
		{Geureomyeon geu eoreun -eun maeu chaksil -ha -n cheongnyeon -eul al -ge doe -n geos -eul mopsi gippeoha -et -da.}
		{그러면 그 어른 -은 매우 착실 -하 -ㄴ 청년 -을 알 -게 되 -ㄴ 것 -을 몹시 기뻐하 -았 -다.}
		{then	\textsc{def}	adult	\textsc{top}	very	steadfastness	\textsc{vbz}	\textsc{rel.pst}	youth	\textsc{acc}	know	\textsc{comp}	become	\textsc{rel.pst}	thing	\textsc{acc}	greatly	rejoice	\textsc{pst}	\textsc{ind}}
		{Then the grown-up would greatly rejoice in getting to know a very steady-going young man.}

\pagebreak
\end{itemize}